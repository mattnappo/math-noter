\documentclass[12pt]{article}
\usepackage[margin=0.80in]{geometry}

\usepackage{mdframed}
\usepackage{amsmath}
\usepackage{amsfonts}
\usepackage{amsthm}
\usepackage{amssymb}
\usepackage{enumerate}
\usepackage{enumitem}
\usepackage{graphicx}
\usepackage[all]{xy}
\usepackage{wrapfig}

\newmdtheoremenv{theorem}{Theorem}
\newmdtheoremenv{definition}{Definition}
\newtheorem{example}{Example}

\title{Notes}
\author{Author}
\date{\today}

\begin{document}

\maketitle

\[
\Delta (\hat{\gamma})^2 = \lambda \frac{\partial \gamma}{\partial t}
\]

Now consider a topological space \((X, \tau)\) where \(\tau\) is the discrete topology \(\tau_{\text{discrete}}\). This is something that has some words here and there are some words that are like I can type here asd asd asd asd asd asd and some math like the square root of \(\pi\) is equal to the integral from \(-\infty\) to \(\infty\) of the Gaussian function (in full form).

\begin{theorem}
\(\tau_{\text{discrete}}\) has \(n\) elements and \(|G|\) is \(\langle a \rangle\)
\end{theorem}

\begin{proof}
The proof is quite simple actually.
\end{proof}

\begin{definition}
A topology is a set \(\tau\) and set \(X\) so that the following axioms hold:
\begin{enumerate}
\item[\textbf{O1:}] Closed
\item[\textbf{O2:}] Null set in \(X\)
\end{enumerate}
\end{definition}

\begin{example}
The trivial topology is just the set \(\tau = \{\varnothing, X\}\)
\end{example}

\end{document}
