\documentclass[12pt]{article}
\usepackage[margin=0.80in]{geometry}

\usepackage{amsmath}
\usepackage{amsfonts}
\usepackage{amsthm}
\usepackage{amssymb}
\usepackage{enumerate}
\usepackage{enumitem}
\usepackage{graphicx}
\usepackage[all]{xy}
\usepackage{wrapfig}

\newtheorem{theorem}{Theorem}
\newtheorem{definition}{Definition}
\newtheorem{example}{Example}

\title{Notes}
\author{Author}
\date{\today}

\begin{document}

\maketitle

\[\Delta (\hat{\gamma})^2 = \lambda \frac{\partial \gamma}{\partial t}\]

Consider a topological space \((X, \tau)\) where \(\tau\) is the discrete topology \(\tau_{\text{discrete}}\). Lorem ipsum dolor sit amet, consectetur adipiscing elit. Sed vitae semper nunc. Donec auctor, nisl id lacinia tincidunt, nunc nunc tincidunt nunc, id lacinia nunc nunc nec nunc. 

The square root of \(\pi\) is equal to the integral from \(-\infty\) to \(\infty\) of the Gaussian function (in full form).

\begin{theorem}
\(\tau_{\text{discrete}}\) has \(n\) elements and \(|G|\) is \(<a>\)
\end{theorem}

\begin{proof}
The proof is quite simple actually.
\end{proof}

\begin{definition}
A topology is a set \(\tau\) and set \(X\) such that the following axioms hold:
\begin{enumerate}
\item[\textbf{O1:}] Closed
\item[\textbf{O2:}] Null set in \(X\)
\end{enumerate}
\end{definition}

\begin{example}
The trivial topology is just the set \(\tau = \{\text{null set}, X\}\)
\end{example}

\end{document}
